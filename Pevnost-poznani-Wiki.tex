\documentclass[hyperref=unicode, presentation,10pt]{beamer}

\usepackage[absolute,overlay]{textpos}
\usepackage{array}
\usepackage{graphicx}
\usepackage{adjustbox}
\usepackage[version=4]{mhchem}
\usepackage{chemfig}
\usepackage[utf8]{inputenc}
\usepackage{caption}

\addtobeamertemplate{frametitle}{
	\let\insertframetitle\insertsectionhead}{}
\addtobeamertemplate{frametitle}{
	\let\insertframesubtitle\insertsubsectionhead}{}

\makeatletter
\CheckCommand*\beamer@checkframetitle{\@ifnextchar\bgroup\beamer@inlineframetitle{}}
\renewcommand*\beamer@checkframetitle{\global\let\beamer@frametitle\relax\@ifnextchar\bgroup\beamer@inlineframetitle{}}
\makeatother
\setbeamercolor{section in toc}{fg=red}
\setbeamertemplate{section in toc shaded}[default][100]

\usepackage{fontspec}
\usepackage{unicode-math}

\usepackage{polyglossia}
\setdefaultlanguage{czech}

\def\uv#1{„#1“}

\mode<presentation>{\usetheme{default}}
\usecolortheme{crane}

\setbeamertemplate{footline}[frame number]

\makeatletter
\setbeamertemplate{navigation symbols}{}


\title{Wikipedie a chemie}

\subtitle{Jak využít Wikipedii ve výuce}
\author{Zdeněk Hugo Moravec \\ zdenek.moravec@wikimedia.cz \\ \adjincludegraphics[height=55mm]{img/IUPAC_PSP.jpg}}

\date{}

\begin{document}

\begin{frame}
	\titlepage
\end{frame}

\section{Wikipedie a chemie}
\frame{
	\frametitle{}
	\vfill
	\begin{columns}
		\begin{column}{.75\textwidth}
			\begin{itemize}
				\item Článků s \href{https://cs.wikipedia.org/wiki/Kategorie:Chemie}{\underline{chemickou tematikou}} je na české Wikipedii poměrně dost, ale některé mají \href{https://cs.wikipedia.org/wiki/Wikipedie:WikiProjekt_V\%C4\%9Bda\%2FTipy_na_\%C4\%8Dl\%C3\%A1nky}{\underline{nízkou kvalitu}}.
				\item Velká část nových článků vzniká překladem, nejčastěji z angličtiny nebo němčiny.
				\item Editory se zájmem o chemii sdružuje \href{https://cs.wikipedia.org/wiki/Port\%C3\%A1l:Chemie}{\underline{Portál:Chemie}}.
			\end{itemize}
		\end{column}
		
		\begin{column}{.3\textwidth}
			\begin{figure}
				\adjincludegraphics[width=\textwidth]{img/intro.png}
				\caption*{Rozklad peroxidu vodíku.\footnote[frame]{\href{https://commons.wikimedia.org/wiki/File:Katalyticky.rozklad.peroxidu.vodiku.png}{Zdroj: Knappová Tereza/Commons}}}
			\end{figure}
		\end{column}
	\end{columns}
	\vfill
}

\section{Středoškolská Odborná Činnost}
\frame{
	\frametitle{}
	\vfill
	\begin{itemize}
		\item V rámci SOČ se studenty vytváříme databázi chemických pokusů.
		\item Práce se skládá z několika kroků:
		\begin{enumerate}
			\item Práce v laboratoři -- pořízení audiovizuálních materiálů
			\item Zpracování získaných souborů
			\item Zpracování doprovodného textu -- teorie, bezpečnost, postup, závěr
			\item Publikace vytvořených materiálů na \href{https://cs.wikibooks.org/wiki/Soubor:Potassium_chlorate_and_gummi_bear_experiment.webm}{\underline{Wikimedia Commons}} a \href{https://cs.wikibooks.org/wiki/Chemick\%C3\%A9_pokusy}{\underline{Wikiknihách}}.
		\end{enumerate}		
	\end{itemize}
	
	\begin{figure}
		\adjincludegraphics[height=.5\textheight]{img/nataceni.jpg}
	\end{figure}
	\vfill
}

\section{Bakalářské a magisterské práce}
\frame{
	\frametitle{}
	\begin{columns}
		\begin{column}{.75\textwidth}
			\vfill
			\begin{itemize}
				\item Se studenty bakalářského a magisterského připravujeme výukové texty na téma \href{https://cs.wikibooks.org/wiki/Anorganick\%C3\%A1_chemie}{\underline{Anorganická chemie}}.
				\item Součástí práce je i tvorba videí souvisejících s tématem práce.
				\item Videa pořizujeme pomocí programu OBS Studio a stříháme převážně pomocí SW ShotCut.
				\item Ukázky videí:
				\begin{itemize}
					\item \href{https://commons.wikimedia.org/wiki/File:Argentometric_titration.webm}{\underline{Argentometrie}}
					\item \href{https://commons.wikimedia.org/wiki/File:Oxodiperoxochromium-complex.webm}{\underline{Peroxokomplex chromu}}
				\end{itemize}
			\end{itemize}
			\vfill
		\end{column}
		
		\begin{column}{.3\textwidth}
			\begin{figure}
				\begin{figure}
					\adjincludegraphics[width=\textwidth]{img/Peroxokomplex.jpg}
				\end{figure}
			\end{figure}
		\end{column}
	\end{columns}
}

\section{Využití Wikipedie při tvorbě výukových materiálů}
\frame{
	\frametitle{}
	\begin{columns}
		\begin{column}{.75\textwidth}
			\vfill
			\begin{itemize}
				\item Rozcestník informací
				\item \href{https://commons.wikimedia.org/}{\underline{Wikimedia Commons}} -- zdroj obrázků, videí a dalších multimediálních souborů
				\begin{itemize}
					\item \href{https://commons.wikimedia.org/wiki/Category:Samples_of_inorganic_compounds}{\underline{Samples of inorganic compounds}}
					\item \href{https://commons.wikimedia.org/wiki/Category:Crystal_structures}{\underline{Crystal structures}}
					\item \href{https://commons.wikimedia.org/wiki/Category:Videos_of_chemistry}{\underline{Videos of chemistry}}
				\end{itemize}
				\item \href{https://www.wikidata.org/}{\underline{Wikidata}} -- hlavní úložiště strukturovaných dat sesterských projektů nadace Wikimedia včetně Wikipedie, Wikicest, Wikizdrojů a dalších
				\item Pokud narazíte na chybu nebo zastaralou informaci, \href{https://cs.wikipedia.org/wiki/Wikipedie:Editujte_s_odvahou}{\underline{nezapomeňte ji opravit}}!
			\end{itemize}
			\vfill
		\end{column}
		
		\begin{column}{.3\textwidth}
			\begin{figure}
				\begin{figure}
					\adjincludegraphics[width=\textwidth]{img/Commons-logo.png}
				\end{figure}				
				\begin{figure}
					\adjincludegraphics[width=\textwidth]{img/Wikidata.png}
				\end{figure}
			\end{figure}
		\end{column}
	\end{columns}
}

\section{Závěr}
\frame{
	\frametitle{}
	\vfill
	\centering \Huge
	\textbf{Děkuji za pozornost} \\[2ex]

	\large
	Zdeněk Hugo Moravec\\
	zdenek.moravec@wikimedia.cz \\
	\vfill
}

\end{document}
